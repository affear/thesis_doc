%!TEX root = thesis.tex

Il viaggio intrapreso verso il conseguimento della laurea è stato certamente complesso. E' stato un percorso formativo lungo e tortuoso, costellato da piccole e grandi difficoltà, quanto da piccole e grandi soddisfazioni. Questi anni mi hanno fatto crescere sia sul piano delle mie competenze e conoscenze ingegneristiche, sia su quello più intimo e personale. Ho affrontato momenti di sconforto, momenti nei quali mi sono chiesto ``ma sto andando davvero nella direzione giusta?'' e anche ``è davvero questo quello che voglio fare?''. L'università mi ha messo a confronto non tanto con l'algebra, la fisica, l'analisi matematica o l'informatica, quanto con me stesso. Credo sia accaduto ad ogni studente, infatti, di accorgersi dopo qualche anno che lo studio ti mette alla prova, ti porta a metterti in gioco in prima persona, ti costringe a rinunce e sacrifici, ti porta a compiere numerose scelte, ti regala vittorie personali quanto, purtroppo, fallimenti. Ci si rende ben presto conto che, per superare queste prove, non servono soltanto intelligenza e diligenza allo studio (qualità certamente necessarie), ma anche buone dosi di determinazione e di ambizione. E' inevitabile e doveroso, infatti, il moto di stima e rispetto che provo per tutti i miei colleghi che hanno portato a termine con successo i loro studi. Complimenti a tutti voi.  

Durante il viaggio che porta alla laurea, i momenti di sconforto e le dure prove sono accompagnati, fortunatamente, da momenti di estrema gioia e realizzazione in cui tutto il sudore buttato sui libri (o sul portatile) viene ripagato. Nel mio caso, sono stati questi i momenti in cui ho capito di essere nel posto giusto e che ne valeva davvero la pena.

Ciò che accomuna, a mio parere, i momenti di paura e sconforto quanto i momenti di grande realizzazione e gioia, sono le persone. Credo fortemente, infatti, che nessuno riuscirebbe ad arrivare alla laurea senza alle spalle un grande team di supporto. Per mia grande fortuna, infatti, non sono stato solo ad affrontare questo percorso. Colgo quindi l'opportunità in questo piccolo spazio all'interno della nostra tesi per ringraziare chi, in piccola o grande misura, mi ha lasciato qualcosa dentro e mi ha aiutato in questi ultimi cinque anni.

Innanzitutto un particolare ringraziamento non va ad una persona, quanto ad una instituzione. Ringrazio vivamente il Politecnico di Milano. Un meravaglioso ateneo, che, grazie all'estrema competenza e distinzione dei suoi docenti, è stato capace di stimolare la mia mente e di plasmarla come quella di un vero ingegnere. L'orgoglio che provo ora per essere stato formato da questo grande ateneo non ha davvero prezzo. Ancora, un grazie enorme.

Secondo, ma non meno importante, al mio (nostro) relatore, Sam Guinea, che con la sua affabilità, disponibilità ed esperienza ci ha trasmesso insegnamenti importanti e ci ha dato la possibilità di sfruttare opportunità davvero uniche.

Grazie a Giacomo Bresciani, il mio ``socio'' di tesi, colui il quale mi è sempre stato vicino in questi ultimi sette mesi. Più che un collega sei stato davvero un amico e mi hai aiutato a ``tirare avanti'' in momenti davvero bui. Grazie.

Un grazie particolare va al mio vero team di supporto, la mia famiglia. Una madre, un padre e una sorella che tutti inviederebbero. Mi avete dato amore e supporto in tutti i sensi e mi avete sempre fatto sentire orgoglioso di me stesso. Grazie a voi.

Grazie, in generale, a tutta la mia famiglia. Tanto la parte malnatese quanto quelle comasca. A tutte le zie e zii e alla nonna. Grazie anche a Giusy e alle sue preghiere. A tutti voi un enorme grazie per il supporto e l'affetto.

Grazie Massimiliano. Un amico che ti fa capire il senso di ``chi trova un amico, trova un tesoro''. Ci sei sempre stato, anche quando io non ci sono stato per te, senza se e senza ma. Senza chiedere. Sei un vero amico. Ringraziarti su queste pagine non è abbastanza, lo so, ma comunque sento il dovere di farlo. Grazie.

Un grazie particolare va anche a Martina. Mi hai aiutato, soprattutto negli ultimi passi, a non cadere in un baratro. Mi hai dato prova di avere un grande cuore e una grande forza. Mi hai fatto capire cosa voglia dire affrontare i problemi e rimboccarsi le maniche. Sei una grande donna e meriti il meglio. Grazie.

Grazie a te Vanessa. Per essermi stata vicina, aver sempre creduto in me e per avermi supportato e sopportato durante la laurea magistrale, un momento particolarmente duro.

Un grazie va anche a tutti i miei amici, i quali mi hanno aiutato non tanto nel mio percorso universitario, quanto a deviare un poco ogni tanto e, quindi, a mantenere la mia sanità mentale. Elenco qui i loro soprannomi anche se tutti loro hanno un piccolo spazio nel mio cuore. Grazie Robi, Fox, Zebra, Colo, Geno, Teo, Tia e Alep.

Un enorme ringraziamento va ai miei compagni di percorso, i miei colleghi, con cui ho davvero condiviso le difficoltà e le gioie. Grazie Alessandro Coltro e Nicolò Bernaschina, grandi amici oltre che colleghi. Grazie a Luca Borghese, il ``capo'', e ad Andrea Bernardini per il loro grande genio, la loro disponibilità e la loro infinita umiltà. Mi avete davvero insegnato qualcosa. Grazie.

Da aggiungere alla lista dei miei colleghi sono Fabio Arcidiacono e Riccardo Tommasini. Siete meritevoli di un paragrafo tutto vostro, perché, alla fine, ``nobody works like us''.