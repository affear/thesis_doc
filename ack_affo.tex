%!TEX root = thesis.tex

Il percorso che porta ogni studente al conseguimento di una laurea è complesso, tortuoso e costellato da piccole e grandi difficoltà. Durante questo percorso si affrontano momenti di sconforto, momenti nei quali capita di chiedersi ``ma sto andando davvero nella direzione giusta?'', oppure ``è davvero questo quello che voglio fare?''. Ogni studente si trova spesso ad affrontare questi momenti e deve, se vuole proseguire nel suo percorso, superarli. Per questo motivo la laurea, a mio personale parere, è più una questione di determinazione e ambizione, piuttosto che di intelligenza o fortuna. Quanti sono stati i momenti in cui avrei voluto gettare la spugna, eppure eccomi qui.

Ai momenti di grande difficoltà e sconforto si contrappongono, ovviamente, momenti di estrema felicità in cui tutto il sudore buttato sui libri (o sul portatile) viene ripagato. Sono questi i momenti che ti fanno davvero capire che sei nel posto giusto e che ne è valsa davvero la pena.

Ciò che accomuna i momenti di paura e sconforto quanto i momenti di grande realizzazione e gioia, sono le persone. Nessuno riuscirebbe ad arrivare alla laurea senza alle spalle un grande team di supporto. Per mia grande fortuna, infatti, non sono stato solo ad affrontare questo percorso. Colgo quindi l'opportunità in questo piccolo spazio all'interno della tesi per ringraziare chi, in piccola o grande misura, mi ha lasciato qualcosa dentro e mi ha aiutato in questi ultimi cinque anni.

Innanzitutto un particolare ringraziamento non va ad una persona, quanto ad una instituzione. Ringrazio vivamente il Politecnico di Milano. Un meravaglioso ateneo, che, grazie all'estrema competenza e distinzione dei suoi docenti, è stato capace di stimolare la mia mente e di formarla come quella di un vero ingegnere. L'orgoglio che provo ora per essere stato formato da questo grande ateneo non ha davvero prezzo. Ancora, un grazie enorme, Politecnico di Milano.

Secondo, ma non meno importante, al mio (nostro) relatore, Sam Guinea, che con la sua affabilità, disponibilità ed esperienza ci ha trasmesso molto e ci ha dato la possibilità di sfruttare opportunità davvero uniche.

Grazie a Giacomo Bresciani, il mio ``socio'' di tesi, colui il quale mi è sempre stato vicino in questi ultimi sette mesi. Più che un collega sei stato davvero un amico e mi hai aiutato a ``tirare avanti'' in momenti davvero bui. Grazie.

Un grazie particolare va al mio vero team di supporto, la mia famiglia. Una madre, un padre e una sorella che tutti inviederebbero. Mi avete dato amore e supporto in tutti i sensi e mi avete sempre fatto sentire orgoglioso di me stesso. Grazie a voi.

Grazie, in generale, a tutta la mia famiglia. Tanto la parte malnatese quanto quelle comasca. A tutte le zie e zii e alla nonna. Grazie anche a Giusy e alle sue preghiere. A tutti voi un enorme grazie per il supporto e l'affetto.

Grazie Massimiliano. Un amico che ti fa capire il senso di ``chi trova un amico, trova un tesoro''. Ci sei sempre stato, anche quando io non ci sono stato per te, senza se e senza ma. Senza chiedere. Sei un vero amico. Ringraziarti su queste pagine non è abbastanza, lo so, ma comunque sento il dovere di farlo. Grazie.

Un grazie particolare va anche a Martina. Mi hai aiutato, soprattutto negli ultimi passi, a non cadere in un baratro. Grazie.

Grazie a te Vanessa. Per essermi stata vicina, aver sempre creduto in me e per avermi supportato e sopportato durante la laurea magistrale, un momento particolarmente duro (in senso accademico).

Un grazie va anche a tutti i miei amici, i quali mi hanno aiutato non tanto nel mio percorso universitario, quanto a deviare un poco ogni tanto e, quindi, a mantenere la mia sanità mentale. Elenco qui i loro soprannomi (molto più significativi dei loro nomi e cognomi) anche se tutti loro hanno un piccolo spazio nel mio cuore. Grazie Robi, Fox, Zebra, Colo, Geno, Teo, Tia e Alep.

Un enorme ringraziamento va ai miei compagni di percorso, i miei colleghi, con cui ho davvero condiviso le difficoltà e le gioie. Grazie Alessandro Coltro e Nicolò Bernaschina, grandi amici oltre che colleghi. Grazie a Luca borghese, il ``capo'', per il suo grande genio, la sua disponibilità e la sua infinita umiltà. Mi hai davvero insegnato qualcosa Luca. Grazie.

Da aggiungere alla lista dei miei colleghi sono Fabio Arcidiacono e Riccardo Tommasini. Siete meritevoli di un paragrafo tutto vostro, perché, alla fine, ``nobody works like us''.