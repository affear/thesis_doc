%!TEX root = ../thesis.tex
%%--------------------------------------------------------------------------
%% INTRODUCTION
%%--------------------------------------------------------------------------

The problem of resources allocation, in the past years, has increasingly attracted the attention of the community for its important implications in the field of energy saving. The rapid growth of cloud services has in fact heavily raised the number of data centers all over the world, consequentially increasing their power consumption. Nowadays, data centers energy consumption has indeed become a really urgent and important problem as the power they need has reached the \% of the world's total in \todo{\% year and reference}.\\
For that reason, an intelligent and efficient strategy for resources allocation is critical to try to make the most of available hardware. Within a IaaS, one way to achieve this goal is try to have the minimum number of running servers while maintaining all the virtual machines which were requested from the users running and available. This is commonly done through an intelligent \textit{VMs Placement} on available servers when a VM is requested. Exploiting this solution it is possible to ensure that the data center is ``filled'' in a consistent way, avoiding under-allocated resources.\\
The problem with the previous solution is that it doesn't cover those cases in which VMs are deallocated from the hosting hardware. In these cases, in fact, the system could reach a state where all the data center servers are no more used efficiently, leaving some of them, for example, under-utilized, and so consuming more power than needed. To address this problem it is possible to \textit{consolidate} the arrangement of VMs within the data center, migrating them from under-utilized servers to servers which can host them; thanks to this migrations it will be possible to ``empty'' these under-utilized machines and take them in an energy-saving power state, such as deep-sleep. This process is called \textit{VMs Consolidation} and it is  very interesting way to increase power saving. In fact, as deepen in the State of the Art chapter (see \ref{chap:sota}), during the past years VMs Consolidation has gained more and more attention from the community, and a lot of algorithms and technique were proposed to address it in the best way.\\
However, despite it is one of the more interesting approach to power saving, VMs Consolidation lacks of implementations, especially in non-proprietary IaaS , and the majority of the solutions are mostly theoretical without practical tests in real environments. For example OpenStack (see chapter \ref{chap:openstack_devstack}), the most important and used IaaS open-source solution doesn't provide an official implementation or a way to exploit VMs consolidation.

 
At the beginning of the development of the thesis we were mainly focused on the implementation of a module for OpenStack that would allow us to implement different consolidation algorithms and to test them to see their impact on a real cloud system, in terms of resource allocation. At the beginning we faced the problem of running, testing and benchmarking our code in an OpenStack environment. Indeed to deal with aspects like Scheduling, Virtual Machines Placement, and Server Consolidation we needed an highly configurable system that would allow us to run simulations and benchmarks to evaluate the soundness of our solutions.\\ 
A common barrier to experimenting with cloud infrastructure is in fact the lack of access to a fully functional cloud installation. Although OpenStack can be used to create testbeds, it is not uncommon in literature to find works that are plagued by unrealistic setups that use only a handful of servers. Moreover, setting up a testbed is necessary but not sufficient. One must also be able to create repeatable experiments that can be used to compare one’s results to baseline or related approaches from the state of the art.
So we designed and develop a system to address this problem: we wanted it to be fully customizable to match different requirements and let the user customize a lot of aspects, such as the structure of the environments, the number of Compute Nodes (the nodes that host Virtual Machines), their fake characteristics (see section \ref{sec:openstack_fake_drivers} on \code{FakeDrivers}) or the OpenStack services to run. Secondly we needed a way to automatically simulate, in a repeatable way, the workload generated from user applications that normally run on an OpenStack installation. At last we realized that it would be very useful to show the real time data of the simulations to analyze the behavior of the system in different configurations.
Therefore we decided to develop aDock, a suite of tools for creating performance, sandboxed, and configurable cloud infrastructure experimentation environments that developer, sysadmins and researchers can exploit to access a fully functional cloud installation of OpenStack.
