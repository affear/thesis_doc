%!TEX root = thesis.tex
\begin{spacing}{1.1}
Quando si scrive codice per una piattaforma software di cloud computing, il \emph{testing} è di primaria importanza. Lo sviluppatore può utilizzare i test di unità e di integrazione per valutare la bontà del suo codice. Tuttavia, i software cloud possono essere molto complessi. Talvolta, inoltre, lo sviluppatore vorrebbe scrivere del codice che influenzi il comportamento dell'intero sistema e non di un solo componente. In questi casi il semplice \emph{testing} non è sufficiente: lo sviluppatore ha bisogno di sapere come il nuovo codice si comporterà all'interno del sistema. E' quindi necessario eseguire simulazioni complesse ed analizzare i risultati ottenuti. Tuttavia, può essere difficile per lo sviluppatore, spesso sprovvisto di grandi quantità di risorse hardware, sperimentare codice in un intero sistema cloud, il quale è solitamente ospitato da decine di macchine fisiche. La virtualizzazione del sistema è quindi d'obbligo.

La nostra soluzione è aDock, un sistema modulare che sfrutta Docker per le sue tecniche di ``virtualizzazione leggera'' e OpenStack come software open-source di cloud computing di riferimento. aDock, in questo modo, consente agli utenti di avviare sistemi cloud estremamente leggeri, eseguire simulazioni, salvare i dati in uscita e mostrarli in tempo reale grazie ad un'interfaccia utente.

Come ulteriore contributo è stato sviluppato un servizio di consolidamento di macchine virtuali per OpenStack testato con quattro diversi algoritmi di consolidamento. Il consolidamento è, ad oggi, uno dei temi di primaria importanza nei sistemi cloud. Si tratta dello sviluppo di strategie per l'allocazione delle risorse che garantiscano l'utilizzo ottimale dell'hardware disponibile ai fini del risparmio energetico, problema sempre più rilevante nei \emph{data center}, data la loro incessante crescita.

Al fine di valutare la nostra soluzione abbiamo usato aDock per lanciare un sistema OpenStack e confrontare gli algoritmi di consolidamento proposti. Grazie ad aDock siamo stati in grado di avviare il sistema in un tempo ragionevole, anche su computer portatili. I risultati delle simulazioni hanno dimostrato che i quattro algoritmi di consolidamento portano ad un notevole miglioramento nell'allocazione delle risorse del sistema.
\end{spacing}