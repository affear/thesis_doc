%!TEX root = thesis.tex
\begin{spacing}{1.1}
Quando si scrive codice per una piattaforma software di cloud computing, il \emph{testing} è di primaria importanza. Lo sviluppatore può utilizzare i test di unità e di integrazione per valutare la bontà del suo codice. Tuttavia, i software cloud possono essere molto complessi. Talvolta, inoltre, si verificano casi in cui lo sviluppatore vorrebbe scrivere del codice che influenzi il comportamento dell'intero sistema e non di un solo componente. In questi casi il semplice \emph{testing} non è sufficiente: lo sviluppatore ha bisogno di sapere come il nuovo codice si comporterà all'interno del sistema. E' quindi necessario eseguire simulazioni complesse ed analizzare i risultati ottenuti. Tuttavia, è difficile per uno sviluppatore, che spesso è provvisto di una limitata quantità di risorse hardware, sperimentare codice in un intero sistema cloud, il quale deve essere ospitato da almeno decine di macchine fisiche. La virtualizzazione del sistema è quindi d'obbligo.

La nostra soluzione è aDock, un sistema modulare che sfrutta le tecniche di ``virtualizzazione leggera'' di Docker e, come software open-source di cloud computing di riferimento, OpenStack per consentire agli utenti di avviare sistemi cloud estremamente leggeri, eseguire simulazioni, salvare i dati in uscita e mostrarli in tempo reale grazie ad un'interfaccia utente.

Come contributo secondario è stato sviluppato un servizio di consolidamento di macchine virtuali per OpenStack testato con quattro diversi algoritmi di consolidamento. Il consolidamento è uno dei temi di primaria importanza ad oggi nei sistemi cloud. Esso consiste in strategie per l'allocazione delle risorse che portino ad utilizzare al meglio l'hardware disponibile e, quindi, a risparmiare energia. Il risparmio energetico, infatti, è diventato un problema critico nei \emph{data center}, dato il loro crescente consumo energetico.

Al fine di valutare la nostra soluzione abbiamo usato aDock per lanciare un sistema OpenStack e confrontare gli algoritmi di consolidamento proposti. Grazie ad aDock siamo stati in grado di avviare il sistema in un tempo ragionevole, anche su computer portatili. I risultati delle simulazioni hanno dimostrato che i quattro algoritmi di consolidamento hanno portato ad un notevole miglioramento nell'allocazione delle risorse del sistema.
\end{spacing}